% !TEX TS-program = XeLaTeX --shell-escape
\documentclass{modern}

\usepackage{lipsum} % Required to insert dummy text into the template

\institute{University of Caen}
\date{\today}
\author{John Smith}
\title{Assignment 1}
\department{Computer Science}
\subject{Programming}

\begin{document}
\thispagestyle{plain}

\makeheader

\section{Python problem}

The Listing \ref{lstpython} shows Python \code{code}.

\begin{lstlisting}[language=Python, caption=Sample Python script with highlighting, label=lstpython]
def f(x):
	s = ("Test", 2 + 3, {'a': 'b'}, x)  # Comment
	if s:
		print(s[0].lower())

class Foo:
	def __init__(self):
		byte_string = 'newline:\n also newline:\x0a'
		text_string = "Cyrillic R is \u042f."
		self.makeSense(whatever=1)
	
	def make_sense(self, whatever):
		self.sense = whatever

x = len('abc')
print(f.__doc__)
\end{lstlisting}

\lipsum[1]

\section{Java problem}

\lipsum[2]

The Figure \ref{exemple} shows an example.

\begin{figure}[H]
\centering
\includegraphics[width=0.5\textwidth]{images/example_figure.png}
\caption{A sample figure}
\label{exemple}
\end{figure}

\lipsum[3]

\begin{lstlisting}[language=Java, caption=Sample Java script]
public class Foo {

    public static void main(String[] args) {
        printOne();
        printOne();
        printTwo();
    }

    public static void printOne() {
        System.out.println("Hello world");
    }
    
    public static void printTwo() {
        printOne();
        printOne();
    }

}
\end{lstlisting}

\usemintedstyle{github}
\begin{minted}{python}
def f(x):
    s = ("Test", 2 + 3, {'a': 'b'}, x)  # Comment
    if s:
        print(s[0].lower())

class Foo:
    def __init__(self):
        byte_string = 'newline:\n also newline:\x0a'
        text_string = "Cyrillic R is \u042f. Oops \u042g"
        self.makeSense(whatever=1)
	
    def make_sense(self, whatever):
        self.sense = whatever

x = len('abc')
print(f.__doc__)
\end{minted}


\lipsum[4]

\end{document}